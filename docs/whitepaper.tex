\documentclass[11pt, a4paper]{article}

% ----------------------------------------------------------------------
% P a c k a g e s
% ----------------------------------------------------------------------

% Encoding & Language
\usepackage[utf8]{inputenc}
\usepackage[T1]{fontenc}
\usepackage[english]{babel} % English language settings

% Font & Typography
\usepackage[sc]{mathpazo} % Palatino Font with Small Caps
\usepackage[scaled=0.95]{helvet} % Helvetica as Sans-Serif
\usepackage{inconsolata} % Monospace Font for Code
\usepackage{microtype} % Improved spacing & typography
\linespread{1.15} % Slightly increased line spacing for readability

% Layout & Geometry
\usepackage[a4paper, top=25mm, bottom=25mm, left=25mm, right=25mm]{geometry}
\usepackage{parskip} % Separate paragraphs with space instead of indentation
\usepackage{float}

% Graphics & Colors
\usepackage{graphicx}
\usepackage{xcolor}
\usepackage{tikz}
\usetikzlibrary{shapes,arrows,positioning,calc}

\definecolor{primaryBlue}{RGB}{0, 51, 102} % Dark blue for headings
\definecolor{accentColor}{RGB}{50, 150, 200}
\definecolor{codeBg}{RGB}{245, 245, 245}

% Code Listings
\usepackage{listings}
\lstdefinelanguage{Rust}{
    keywords={true, false, unsafe, async, await, move, use, pub, mod, let, mut, fn, struct, impl, enum, type, crate, match, if, else, loop, break, continue, return, for, in, while, where, trait, dyn},
    keywordstyle=\color{primaryBlue}\bfseries,
    ndkeywords={String, Option, Result, Box, Vec, Arc, Rc, RefCell, Mutex, AtomicBool, u8, u16, u32, u64, usize, i8, i16, i32, i64, isize, f32, f64, bool, char, str, Self, DashMap, Ipv4Addr, Ipv6Addr, BytesMut},
    ndkeywordstyle=\color{accentColor}\bfseries,
    comment=[l]{//},
    morecomment=[s]{/*}{*/},
    commentstyle=\color{green!50!black}\itshape,
    string=[b]",
    stringstyle=\color{red!60!black},
    sensitive=true
}

\lstset{
    language=Rust,
    backgroundcolor=\color{codeBg},
    basicstyle=\ttfamily\small,
    breaklines=true,
    frame=single,
    rulecolor=\color{gray!20},
    keywordstyle=\color{primaryBlue}\bfseries,
    commentstyle=\color{green!50!black}\itshape,
    stringstyle=\color{red!60!black},
    numbers=left,
    numberstyle=\tiny\color{gray},
    stepnumber=1,
    numbersep=5pt,
    showstringspaces=false
}

% Headings
\usepackage{titlesec}
\titleformat{\section}{\Large\bfseries\sffamily\color{primaryBlue}}{\thesection}{1em}{}
\titleformat{\subsection}{\large\bfseries\sffamily\color{primaryBlue}}{\thesubsection}{1em}{}
\titleformat{\subsubsection}{\normalsize\bfseries\sffamily\color{primaryBlue}}{\thesubsubsection}{1em}{}

% Header & Footer
\usepackage{fancyhdr}
\pagestyle{fancy}
\fancyhf{}
\fancyhead[L]{\sffamily\small Mavi VPN - Technical Whitepaper}
\fancyhead[R]{\sffamily\small Confidential}
\fancyfoot[C]{\thepage}
\renewcommand{\headrulewidth}{0.4pt}
\renewcommand{\footrulewidth}{0.0pt}

% Hyperlinks
\usepackage[colorlinks=true, linkcolor=primaryBlue, urlcolor=accentColor, citecolor=primaryBlue, pdfborder={0 0 0}]{hyperref}

% ----------------------------------------------------------------------
% D o c u m e n t
% ----------------------------------------------------------------------

\begin{document}

% --- Title Page ---
\begin{titlepage}
    \centering
    \vspace*{3cm}
    
    {\scshape\Large Technical Architecture Document \& Whitepaper \par}
    \vspace{0.5cm}
    {\Huge\bfseries\sffamily\color{primaryBlue} Mavi VPN \par}
    \vspace{0.5cm}
    {\large\itshape Next-Generation Privacy Architecture \& Rust Implementation \par}
    
    \vspace{3cm}
    
    \textbf{\sffamily Executive Summary}
    
    \vspace{0.3cm}
    \begin{minipage}{0.85\textwidth}
        \centering\itshape
        Mavi VPN redefines the standard for modern VPN technologies. By moving away from outdated protocols like OpenVPN and IPsec to a pure \textbf{Rust-based implementation based on IETF QUIC}, Mavi VPN offers unmatched security guarantees and latency values.
        
        Core features include complete memory safety through Rust, censorship resistance through traffic obfuscation (HTTP/3 mimicry), and congestion control optimized for mobile networks (BBR). This document, intended for security researchers, network architects, and auditors, discloses the complete technical functionality from the byte level of packets to the high-level architecture.
    \end{minipage}
    
    \vfill
    
    \begin{tabular}{ll}
        \textbf{Version:} & 2.1.0 \\
        \textbf{Date:} & \today \\
        \textbf{Status:} & Production / Audit-Ready \\
        \textbf{Technology:} & Tokio, Quinn, Rustls, Ring \\
        \textbf{Source Code:} & \url{https://github.com/zerox80/mavi-vpn} \\
    \end{tabular}
\end{titlepage}

% --- Table of Contents ---
\thispagestyle{empty}
\tableofcontents
\newpage

% --- Main Content ---

\section{Introduction}
The internet is increasingly based on mobile devices and unstable networks. Classic VPN protocols, developed in the 90s or 00s (like IPsec or OpenVPN), are optimized for static lines and show serious weaknesses in modern environments:
\begin{enumerate}
    \item \textbf{TCP-in-TCP Meltdown:} When packet loss occurs, cascading retransmissions lead to a throughput collapse.
    \item \textbf{High Overhead:} Outdated handshakes and unnecessary headers waste bandwidth and battery.
    \item \textbf{Lack of Mobility:} Switching from WiFi to LTE often requires a complete re-handshake (multiple seconds of interruption).
\end{enumerate}
Mavi VPN solves these problems through a complete redesign of the architecture using QUIC (Quick UDP Internet Connections).

\section{Threat Model & Security Goals}
Before discussing the implementation, we define the attacker model against which Mavi VPN has been hardened.

\subsection{Attacker Capabilities}
We assume a globally acting, passive observer as well as an active network attacker (e.g., state firewall, ISP, local hacker).
\begin{itemize}
    \item \textbf{DPI (Deep Packet Inspection):} The attacker can read all packets and analyze them statistically.
    \item \textbf{Active Probing:} The attacker can actively address the server to check if it is a VPN server.
    \item \textbf{Man-in-the-Middle (MitM):} The attacker can forge certificates or compromise CA infrastructure.
\end{itemize}

\subsection{Countermeasures}
\begin{itemize}
    \item \textbf{Against DPI:} Use of TLS 1.3 Encryption and Padding. The protocol looks like HTTP/3 in "Censorship Resistant Mode".
    \item \textbf{Against Active Probing:} The server does not reply to invalid handshakes (wrong token/key) with a TCP RST or VPN error, but emulates an \textbf{Nginx web server} (HTTP 200 OK).
    \item \textbf{Against MitM:} Implementation of \textbf{Certificate Pinning} (SHA-256 Hash Verification). Even a corrupt Root CA cannot decrypt the traffic.
\end{itemize}

\section{Protocol Design & Specification}
Mavi VPN uses QUIC Datagrams for payload transport and QUIC Streams for signaling (Control Plane).

\subsection{Handshake Sequence}
The handshake is minimalist and optimized for 0-RTT/1-RTT.

\begin{enumerate}
    \item \textbf{Client Hello (TLS 1.3):} Contains ALPN \texttt{"mavivpn"} or \texttt{"h3"} (in Stealth Mode).
    \item \textbf{Server Hello:} Certificate exchange.
    \item \textbf{Control Stream Open:} As soon as the QUIC tunnel is established, the client opens a bidirectional stream.
    \item \textbf{Authentication (Client $\to$ Server):} Sending the \texttt{ControlMessage::Auth} packet (Bincode serialized).
    \item \textbf{Configuration (Server $\to$ Client):} On success, the server sends IP, DNS, and routing info (\texttt{ControlMessage::Config}).
\end{enumerate}

\subsection{IP & Address Management}
The backend uses an efficient \textbf{O(1) Allocator} for IP addresses (see \texttt{state.rs}).
\begin{itemize}
    \item \textbf{Data Structure:} A \texttt{Vec<Ipv4Addr>} acts as a stack (LIFO).
    \item \textbf{Initialization:} At startup, all free IPs of the /24 subnet are pushed onto the stack.
    \item \textbf{Assignment:} \texttt{pop()} returns a free IP in constant time.
    \item \textbf{Release (RAII):} An \texttt{IpGuard} struct ensures that IPs automatically return to the pool (\texttt{push()}) as soon as the connection drops (Drop Trait).
\end{itemize}

\section{Technical Architecture (Rust Implementation)}

\subsection{Backend (Async Tokio)}
The server is implemented completely asynchronously.
\begin{itemize}
    \item \textbf{Concurrency:} We use \texttt{DashMap} for thread-safe access to the peer list without global locks.
    \item \textbf{Zero-Copy IO:} Incoming packets from the TUN interface are read using \texttt{BytesMut}. \texttt{split\_to()} allows distributing "slices" of data to individual client tasks without copying memory (\texttt{memcpy} avoidance).
\end{itemize}

\begin{lstlisting}[language=Rust]
// Zero-Copy Pattern in main.rs
let packet = buf.split_to(n).freeze(); 
// 'packet' is now an immutable shared owner of the memory.
// It is sent to the client channel without copying.
if let Err(e) = tx_client.try_send(packet) { ... }
\end{lstlisting}

\subsection{Android Client (Hybrid JNI)}
To bypass Java/Kotlin performance issues, the entire network stack runs natively.
\begin{itemize}
    \item \textbf{VpnService:} Android creates the TUN interface.
    \item \textbf{File Descriptor Transfer:} The FD of the TUN interface is passed to Rust via JNI.
    \item \textbf{Tokio Runtime:} A dedicated runtime within the app processes packets.
    \item \textbf{Battery Saving:} Through asynchronous IO (Epoll/Kqueue), the thread sleeps when no packets are pending instead of remaining in busy loops.
\end{itemize}

\section{Network Engineering: MTU & MSS}

\subsection{The Fragmentation Problem}
VPN tunnels add headers (IP + UDP + QUIC + Encryption). A 1500 byte packet from the PC no longer fits into a 1500 byte packet of the physical network.

\subsection{Mavi VPN Solution Strategy}
We rely on a strict separation of Inner and Outer MTU:
\begin{itemize}
    \item \textbf{Outer MTU: 1500 Bytes.} The physical interface uses standard Ethernet frames.
    \item \textbf{Inner MTU: 1280 Bytes.} This is the guaranteed minimum for IPv6.
    \item \textbf{Overhead Budget:} $1500 - 1280 = 220$ Bytes. This is more than enough for QUIC headers (~20-50 Bytes) and encryption tags.
\end{itemize}

\subsection{MSS Clamping Implementation}
Since Path MTU Discovery (PMTUD) is unreliable on the internet (ICMP Blocking), we manipulate the TCP Handshake ("SYN" packets) directly on the server.
\begin{itemize}
    \item \textbf{IPv4 MSS:} 1240 Bytes. (1280 MTU - 20 IP - 20 TCP).
    \item \textbf{IPv6 MSS:} 1220 Bytes. (1280 MTU - 40 IPv6 - 20 TCP).
\end{itemize}
This is enforced via \texttt{iptables} in the Docker container:
\begin{verbatim}
iptables -t mangle -A FORWARD -p tcp --tcp-flags SYN,RST SYN 
         -j TCPMSS --set-mss 1240
\end{verbatim}

\section{Performance & Congestion Control}

\subsection{BBR (Bottleneck Bandwidth and RTT)}
Mavi VPN uses Google's BBR algorithm instead of CUBIC.
\begin{itemize}
    \item \textbf{Why BBR?} CUBIC interprets packet loss as congestion and halves the rate. In mobile networks, however, packet loss is often stochastic (interference), not congestion-related.
    \item \textbf{Result:} BBR "guesses" the available bandwidth by measuring the RTT and delivers up to 10x higher throughput at 1\% Packet Loss.
\end{itemize}

\subsection{Generic Segmentation Offload (GSO)}
The server activates UDP GSO (\texttt{enable\_segmentation\_offload(true)}). This allows user space to pass 64KB data blocks to the kernel, which are only split into MTU-sized packets by the network card. This massively reduces CPU load (context switches).

\section{Deployment & Operations}

\subsection{Containerization}
The backend is delivered as a Multi-Stage Docker Image.
\begin{itemize}
    \item \textbf{Base:} Debian Bookworm Slim.
    \item \textbf{Builder:} Rust 1.84 (Static Linking).
    \item \textbf{Optimization:} Dummy file caching strategy in Dockerfile accelerates rebuilds by caching dependencies as long as \texttt{Cargo.toml} does not change.
\end{itemize}

\subsection{Scalability}
Due to the event-based design (Event Loop), a single vCPU core can serve approx. 5,000 to 10,000 concurrent clients (bandwidth dependent), as memory per client is minimal (only Channel + Map Entry).

\section{Conclusion}
Mavi VPN represents the state of the art in VPN design. Through the symbiosis of Rust's safety, QUIC's performance, and intelligent network configuration (MSS Clamping, BBR), it offers a robust solution for privacy in hostile networks.

\end{document}
